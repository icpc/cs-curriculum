\documentclass[t, handout]{beamer}

\usepackage[utf8]{inputenc}
\usepackage{minted}
\usepackage{graphicx}

\title{Number Theory: Mods and Primes}
\author{Aditya Arjun, Kevin Chen}
\institute{CS 104C}
\date{Fall 2019}

\setbeamertemplate{footline}[frame number]{}

\setbeamertemplate{navigation symbols}{}

\begin{document}
 
\frame{\titlepage}
 
\begin{frame}

    \frametitle{Modular Arithmetic}

    \begin{itemize}

        \item

        As you've seen in past weeks' homeworks, many problems ask you to output an answer ``modulo $10^9 + 7$''
        
        \item

        We use this number because
        \begin{itemize}
            \item It is prime,
            \pause
            \item It fits into an integer,
        \end{itemize}

        \item How do we work with ``modulo''?
    \end{itemize}

\end{frame}


\begin{frame}
    \frametitle{Division Algorithm.}
    \begin{itemize}
        \item For every pair of integers $a, b$, there exist unique integers $q, r$ such that $a = q b + r$ and $0 \leq r < b$.
        
        \pause

        As an example, if $a = 11, b = 5$, $a = 2 b + 1$. The quotient is $q = 2$, and the remainder is $r = 1$.
        
        \pause 
        
        \item In most languages, $a / b = q$, $a \% b = r$.
        
        \pause
        
        \item Be careful with negative values!
    \end{itemize}
\end{frame}



\begin{frame}
    \frametitle{Modular Arithmetic}
    \begin{columns}
        \begin{column}{0.5\textwidth}
            \begin{center}
             \includegraphics[width=\textwidth]{clock.png}
             \end{center}
        \end{column}
        
        \begin{column}{0.5\textwidth}
        \begin{itemize}
            \item Clocks work ``modulo 12''.
            
            \pause
            
            \item What is 3 hours after 11? \pause 2.
            
            \pause
            
            \item Then we say $11 + 3 \equiv 2 \pmod{12}$.

        \end{itemize}

        \end{column}
    \end{columns}
\end{frame}

\begin{frame}

    \frametitle{Modular rules}
    
    Which of the following are correct?
    \begin{itemize}
        \item $(a + b) \% m = ((a \% m) + (b \% m)) \% m$ ?
        \pause
        \item $(a - b) \% m = ((a \%{m}) - (b \%{m})) \% m$ ?
        \pause
        \item $(a \cdot b) \%{m} = ((a \%{m}) \cdot (b \%{m})) \% m$ ?
        \pause
        \item $(a / b) \%{m} = ((a \%{m}) / (b \%{m})) \% m$ ?
        \pause
        \item $a^b \%{m} = (a \%{m})^b = (a \%{m})^{(b \%{m})} \% m$ ?
    \end{itemize}
    
    \pause
    Answers: First 3 are true, fourth makes little sense, first part of the last one is true.
    
    \pause
    Again, be careful with negatives and mod. We usually implement subtraction as (a \% m - b \% m + m) \% m. 
\end{frame}

\begin{frame}[fragile]
    \frametitle{Definitions}
    
    \begin{itemize}
        \item $d$ is a \textbf{divisor} of $n$ if $d$ divides $n$ evenly.
        \pause
        \item Equivalent: $n \% d = 0$

        \item A number is prime if it has exactly two positive divisors (1 and itself).
    \end{itemize}
    
    \begin{minted}{java}
    boolean isPrime(int n) {
        for (int d = 2; d < n; ++d)
            if (n % d == 0)
                return false;
        return true;
    }
    \end{minted}
    
    \pause
    \begin{itemize}
        \item Time complexity? \pause $O(N)$
        \pause
        \item Can we do better?
    \end{itemize}
    
\end{frame}

\begin{frame}[fragile]

    \frametitle{Primes}
    
    \begin{itemize}
        \item \textbf{Problem:} Given a number $N$, determine whether it is prime
        \item \textbf{Solution:} Check all possible divisors \textbf{up to $\sqrt{N}$}
    \end{itemize}
    
    \begin{minted}{java}
    boolean isPrime(int n) {
        for (int d = 2; d * d <= n; ++d)
            if (n % d == 0)
                return false;
        return true;
    }
    \end{minted}
    
    \pause
    \begin{itemize}
        \item Time complexity? \pause $O(\sqrt{N})$
    \end{itemize}
    
\end{frame}

\begin{frame}[fragile]

    \frametitle{Primes}
    
    \begin{itemize}
        \item \textbf{Problem:} Given a number $N$, get all primes up to $N$
        \pause
        \item \textbf{Solution:} Assume every number is prime. Eliminate all the multiples of $2$, then all the multiples of $3$, \dots
    \end{itemize}
    
    \begin{minted}{java}
    List<Integer> getPrimes(int n) {
        boolean composite = new boolean[n + 1];
        List<Integer> primes = new ArrayList<Integer>();
        for (int i = 2; i <= n; ++i) {
            if (composite[i]) continue;
            primes.add(i);
            for (int j = 2 * i; j <= n; j += i)
                composite[j] = true;
        }
        return primes;
    }
    \end{minted}
    
    \pause
    \begin{itemize}
        \item Time complexity? \pause $O(N \log N)$
        \pause
        \item Can we do better?
    \end{itemize}
    
\end{frame}


\begin{frame}[fragile]

    \frametitle{GCDs}
    
    \begin{itemize}
        \item The greatest common divisor of two numbers $a$ and $b$ is the largest integer $g$ such that both $a$ and $b$ are multiples of $g$.
        
        \pause

        \item Complex Approach: factor both numbers
        
        \pause
        
        \item Euclidean algorithm: Let $a = q b + r$. Note that if some value $g$ divides both $a$ and $b$, then $g$ divides $r$.
        
        \pause

        Proof: $a = q b + r \implies r = a - qb$. Since $g$ divides both $a$ and $b$, $g$ divides $r$.
    \end{itemize}
    
    \pause
    \begin{minted}{java}
    int gcd(int a, int b) {
        if (b == 0) {
            return a;
        }
        return gcd(b, a % b);
    }
    \end{minted}
    
    \pause
    \begin{itemize}
        \item Time complexity? \pause $O(\log N)$
    \end{itemize}
\end{frame}

\begin{frame}[fragile]
    
    \frametitle{Modular rules, again}
    
    \begin{itemize}
        \item \textbf{Problem:} Given $N$ people, calculate how many ways you can make a committee of $M$ people.
        \pause
        \item \textbf{Output your answer modulo} $\mathbf{10^9 + 7}$.
        \pause
        \item How do we compute the answer to this?
    \end{itemize}

\end{frame}

\begin{frame}[fragile]
    
    \frametitle{Modular rules, again}
    
    \begin{itemize}
        \item Recall that we can't divide two numbers modulo a third
        \pause
        \item \textbf{Problem:} Divide two numbers modulo a third \pause (say, for binomial coefficients)
        \pause
        \item Let's say the third number is prime
        \pause
        \item If we want to divide $a$ by $b$ modulo $p$, we'll take $a \cdot b^{-1} \pmod{p}$, where $b^{-1}$ is a ``multiplicative inverse'' of $b$
        \pause
        \item Formally, $b^{-1}$ is a multiplicative inverse of $b$ if $b \cdot b^{-1} \equiv 1 \pmod{p}$
        \pause
        \item We can use Fermat's little theorem: $b^{p-1} \equiv 1 \pmod{p}$
        \pause
        \item Then $b^{-1} = b^{p-2}$
        \pause
        \item \textbf{Problem:} How can we compute this?
        \pause
        \item \textbf{Answer:} Exponentiation by squaring
    \end{itemize}

\end{frame}


\end{document}